\documentclass[11pt,a4paper,english]{article}
\usepackage[english]{babel} % Using babel for hyphenation
\usepackage{lmodern} % Changing the font
\usepackage[utf8]{inputenc}
\usepackage[T1]{fontenc}

%\usepackage[moderate]{savetrees} % [subtle/moderate/extreme] really compact writing
\usepackage{tcolorbox}
\tcbuselibrary{hooks}
\usepackage[parfill]{parskip} % Removes indents
\usepackage{amsmath} % Environment, symbols etc...
\usepackage{amssymb}
\usepackage{framed}
\usepackage{float} % Fixing figure locations
\usepackage{multirow} % For nice tables
%\usepackage{wasysym} % Astrological symbols
\usepackage{graphicx} % For pictures etc...
\usepackage{enumitem} % Points/lists
\usepackage{physics} % Typesetting of mathematical physics examples: 
                     % \bra{}, \ket{}, expval{}
\usepackage{url}
\usepackage{mathtools}
\usepackage{graphicx}
\usepackage{caption}
\usepackage{subcaption}
\newenvironment{algorithm}{%
\refstepcounter{algcounter}
\begin{tcolorbox}
\centerline{Algorithm \thealgcounter}\vspace{2mm}
}
{\end{tcolorbox}}


% To include code(-snippets) with æøå
\usepackage{listings}


\tolerance = 5000 % Bedre tekst
\hbadness = \tolerance
\pretolerance = 2000

\numberwithin{equation}{section}

\definecolor{red}{RGB}{255,20,147}

\newcommand{\conj}[1]{#1^*}
\newcommand{\ve}[1]{\mathbf{#1}} % Vektorer i bold
\let\oldhat\hat
\renewcommand{\hat}[1]{\oldhat{#1}}
\newcommand{\trans}[1]{#1^\top}
\newcommand{\herm}[1]{#1^\dagger}
%\renewcommand{\thefootnote}{\fnsymbol{footnote}} % Gir fotnote-symboler
\newcommand{\Real}{\mathbb{R}}
\newcommand{\bigO}[1]{\mathcal{O}\left( #1 \right)}

%\renewcommand{\thesection}{\Roman{section}} 
%\renewcommand{\thesubsection}{\thesection.\Roman{subsection}}

\newcommand{\spac}{\hspace{5mm}}

\newcounter{algcounter}
\newcommand{\algnum}{\stepcounter{algcounter}\Roman{algcounter}}

\title{MEK4250\\Mandatory assignment 2}
\author{Krister Stræte Karlsen}
\date{\today}

\begin{document}
\tcbset{before app=\parfillskip0pt}
\maketitle

The programs belonging to this assignment is available at \\
\texttt{\color{red}{https://github.com/krikarls/MEK4250}}

\section{Stokes equation}

\subsection{Analysis of the weak formulation (Exercise 7.1)}

In this exercise we want to show some of the properties that must be satisfied in order to have a well posed Stokes problem using a mixed formulation: $u,v \in V = H^1$, $p,q \in Q = L_2 = H_0$.

\textbf{Continuity of $a(u,v)$}

In order to establish \emph{continuity}, or \emph{boundedness}, of $a(u,v)$ we must have that 
\begin{align}
\int \nabla u : \nabla v dx \leq C||u||_1 ||v||_1.
\end{align}
We square both sides
\begin{align}
\left( \int \nabla u : \nabla v dx \right)^2 \leq \left( C||u||_1 ||v||_1 \right)^2
\end{align}
and claim that if (1.2) holds, so does (1.1). Next we apply \emph{Cauchy-Schwartz'}: 
\begin{align*}
\left( \int \nabla u : \nabla v dx \right)^2 &\leq ||\nabla u||_0^2 ||\nabla v||_0^2 \\
&\leq 2||\nabla u||_0^2 ||\nabla v||_0^2 + 2||\nabla u||_0^2 ||v||_0^2 + 2||u||_0^2 ||\nabla v||_0^2 +2||u||_0^2 ||v||_0^2 \\
&= 2 \left( ||u||_0^2 + ||\nabla u||_0^2  \right) \left( ||v||_0^2 + ||\nabla v||_0^2  \right) \\
&= C ||u||_1^2 ||v||_1^2
\end{align*} 
We are now happy and move on to show continuity of $b(u,q)$.

\newpage

\textbf{Continuity of $b(u,q)$}

For continuity of $b(u,q)$ it is required that
\begin{align*}
\int q(\nabla \cdot u) dx \leq C ||u||_1 ||q||_0.
\end{align*}
Using \emph{Cauchy-Schwartz'} 
\begin{align*}
\int q(\nabla \cdot u) dx \leq ||q||_0 ||\nabla \cdot u ||_0. 
\end{align*}
we realize that if  
\begin{align*}
||\nabla \cdot u||_0 \leq  C ||u||_1.
\end{align*}
holds we are on safe ground. Without any worries(since both sides are positive) both sides can be squared. 
\begin{align*}
||\nabla \cdot u||_0^2 &\leq  ||u||_1^2 \\
&=  \left( ||u||_0 + ||\nabla u ||_0  \right) \\
&\leq \left( (D|u|_1)^2 + ||\nabla u||_0^2  \right) \\
&=  \left( (D||\nabla u||_0)^2 + ||\nabla u||_0^2  \right) \\
&= (D^2+1)||\nabla u||_0^2
\end{align*}

It is now sufficient to show that if $||\nabla \cdot u||_0^2 \leq D_2|| \nabla u||_0^2$ holds we have boundedness of $b(u,q)$. We start by writing out both terms(here assuming 2D): 
\begin{align*}
||\nabla \cdot u||_0^2 &= \int \left( \frac{\partial u_1}{\partial x} + \frac{\partial u_2}{\partial y}\right)^2 dx \\ 
|| \nabla u||_0^2 &= \int \left( \frac{\partial u_1}{\partial x} \right)^2 + \left( \frac{\partial u_2}{\partial x} \right)^2 + \left( \frac{\partial u_1}{\partial y} \right)^2 + \left( \frac{\partial u_2}{\partial y} \right)^2 dx
\end{align*}
One can now see that by adding some carefully chosen positive terms to $||\nabla \cdot u||_0^2$ we obtain 
$|| \nabla u||_0^2$, which ensures that $||\nabla \cdot u||_0^2$ must be smaller.
\begin{align*}
||\nabla \cdot u||_0^2 &\leq \int \left( \frac{\partial u_1}{\partial x} + \frac{\partial u_2}{\partial y} \right)^2 + \left( \frac{\partial u_1}{\partial x} - \frac{\partial u_2}{\partial y} \right)^2 + 2\left( \frac{\partial u_2}{\partial x} \right)^2 + 2\left( \frac{\partial u_1}{\partial y} \right)^2  dx \\ 
&= 2 \int \left( \frac{\partial u_1}{\partial x} \right)^2 + \left( \frac{\partial u_2}{\partial x} \right)^2 + \left( \frac{\partial u_1}{\partial y} \right)^2 + \left( \frac{\partial u_2}{\partial y} \right)^2 dx \\
&= D_ 2 || \nabla u||_0^2
\end{align*}


 
\newpage

\textbf{Coersivity of $a(u,q)$}

Coersivity of $a(u,q)$ is established if the following inequality is fulfilled:
\begin{align*}
C||u||_1^2 \leq \int \nabla u : \nabla u dx. 
\end{align*}

We start by rewriting the left hand side and applying \emph{Poincaré}
\begin{align*}
||u||_1^2 &= ||u||_0^2 + ||\nabla u||_0^2 \\
 &\leq (D||\nabla u||_0^2)+||\nabla u||_0^2 \\
 &= (D^2+1)||\nabla u||^2_0 \\
 &=  (D^2+1)\int \nabla u : \nabla u dx
\end{align*}
$C = 1/(D^2+1)$ and we are very happy. 

\newpage

\subsection{Looking into the error estimate (Exercise 7.6)}

To investigate the error estimate
\begin{align*}
||E_u||_1 + ||E_p||_0 \leq Ch^k ||u||_{k+1} + Dh^{l+1}||p||_{l+1}
\end{align*}
we define the \emph{Stokes problem}:
\begin{align*}
-\Delta \mathbf{u} + \nabla p = \mathbf{f}, \quad \mathbf{u} = sin(\pi y) \mathbf{i} + cos(\pi x) \mathbf{j}.
\end{align*}

Computing error and convergence rates for different combinations of polynomials we obtained the results below. 

\begin{figure}[h!] 
\begin{center}
  \includegraphics[scale=0.25]{stokes_loglog.png}
  \end{center}
  \caption{Log-log plot of error for different combinations of polynomials.}
   \label{fig:stokes}
\end{figure}

\begin{table}[H]
\centering
\caption{Convergence rates for $u$ and $p$ respectively. Convergence rates for $u$ are measured in $H^1$ while rates for $p$ are measured in $L_2$.}
\vspace{3mm}
\begin{tabular}{|l|l|l|l|l|l|l|l|l|l|l|}
\hline
 \multicolumn{2}{|c|}{ $P_4-P_3$} &  \multicolumn{2}{|c|}{ $P_4-P_2$} &  \multicolumn{2}{|c|}{ $P_3-P_2$} &  \multicolumn{2}{|c|}{ $P_3-P_1$} \\
\hline
4.29515 & 4.07241 & 2.88045 & 2.91318 & 2.84922 & 2.91507 & 2.08926 & 2.17433 \\
\hline
4.11510 & 4.02557 & 2.96138 & 2.97733 & 2.95217 & 2.97799 & 2.04644 & 2.08797 \\
\hline
4.03899 & 4.00955 & 2.98766 & 2.99487 & 2.98475 & 2.99505 & 2.02451 & 2.04127 \\
\hline
\end{tabular}
\label{tab:vel}
\end{table}


\newpage 

\subsection{Approximation of the shear stress (Exercise 7.7)}

In this section we will have a look at the order of approximation for the \emph{wall shear stress}; that is, for a wall parallel to the $x-axis$ 
\begin{align*}
\tau = \mu \frac{du}{dy}.
\end{align*}

We study this by computing the wall shear stress analytically, then numerically and last compute the error. 

\begin{figure}[h!] 
\begin{center}
  \includegraphics[scale=0.3]{shear_loglog.png}
  \end{center}
  \caption{Log-log plot of the wall shear stress error($L^2$) for different combinations of polynomials.}
   \label{fig:shearstress}
\end{figure}

By looking at the convergence rates(table \ref{tab:shear}) and the log-log plot of the error(figure \ref{fig:shearstress}) one can observe that error behaves like a $H^1-$norm, even though we used a $L^2-$norm to compute the error. That should come as no surprise to the shrewd reader, as the shear stress involves the derivative of the velocity, as does the $H^1-$ norm. In other words; the results for the convergence rates here are the same as in the previous section(table \ref{tab:vel}) where we measured the error in velocity in $H^1$.

\begin{table}[H]
\centering
\caption{Computed convergence rates $r_n$ using a $L^2-$norm for different combinations of polynomials. Increase in $n$ refers to refined mesh. }
\vspace{3mm}
\begin{tabular}{|l|l|l|l|l|l|l|l|l|l|l|}
\hline
 & $P_4-P_3$ & $P_4-P_2$ & $P_3-P_2$ & $P_3-P_1$ \\
\hline
$r_1$ & 4.136169 & 2.950468 & 3.025279 & 2.285654 \\
\hline
$r_2$ & 4.048773 & 2.997513 & 3.080835 & 2.148821 \\
\hline
$r_3$ & 4.021954 & 3.006015 & 3.065985 & 2.054565 \\
\hline
\end{tabular}
\label{tab:shear}
\end{table}

\newpage


\section{Linear elasticity}

To study a phenomenon in numerical linear elasticity called \emph{locking} we look a the following problem:
\begin{align}
-\mu \Delta \mathbf{u} - \lambda \nabla \nabla \cdot \mathbf{u} = \mathbf{f} \quad &in \quad \Omega=(0,1)^2 , \\
\mathbf{u} = \left( \frac{\partial \phi}{\partial u} , -\frac{\partial \phi}{\partial x} \right) \quad &on \quad \partial \Omega ,
\end{align}
where $\phi = sin(\pi x y)$ .

We start deriving and expression for the external forces $\mathbf{f}$.

Since $\nabla \cdot \mathbf{u} = 0$ we have that $\mathbf{f} = - \mathbf{u} \Delta u$. This can be computed using \texttt{sympy} or by hand calculations, like they did in the old days. We find that:
\begin{align*}
\mathbf{f} =  \pi^2(\pi x (x^2+y^2)cos(\pi xy)+ 2y sin(\pi x y)) \mathbf{i} \\ -\pi^2(\pi y(x^2+y^2)cos(\pi xy)+2x sin(\pi xy))\mathbf{j}.
\end{align*}

Problem (2.1)-(2.2) will now be solved using finite element methods in FEniCS. Three different approaches for the $\lambda \nabla \nabla \cdot \mathbf{u}-$term will be tested and discussed. Here listed from stupid to clever:

1) Keep the $\lambda \nabla \nabla \cdot \mathbf{u}$-term as it is. \\
2) Integrate $\lambda \nabla \nabla \cdot \mathbf{u}$ by parts. \\
3) Define $p=\lambda(\nabla \cdot \mathbf{u})$ such that $\lambda \nabla \nabla \cdot \mathbf{u} = \nabla p$ and use a mixed finite element formulation. 

\textbf{Approach \#1}

We think no further, and ignore what we have learned, and implement and solve problem (2.1)-(2.2) in FEniCS as:

\begin{framed}
\texttt{a = inner(grad(u), grad(v))*dx - inner(grad(div(u)),v)*dx \\
L = inner(f,v)*dx}  
\end{framed} 

The results obtained by this approach are very poor and can be viewed in table \ref{tab:err1} and \ref{tab:con1}. 

\begin{table}[H]
\centering
\caption{ L2-errors from using approach \#1 and second order polynomials.}
\vspace{3mm}
\begin{tabular}{|l|l|l|l|l|l|l|l|l|l|l|}
\hline
$\lambda\backslash$N & 8 & 16 & 32 & 64 \\
\hline
1 & 0.014961 & 0.003950 & 0.001002 & 0.000251 \\
\hline
10 & 0.725727 & 0.691354 & 0.026575 & 0.006363 \\
\hline
100 & 3.750028 & 0.606093 & 0.650818 & 2.416408 \\
\hline
1000 & 2.088700 & 0.823344 & 1.165615 & 0.514192 \\
\hline
10000 & 1.426623 & 1.681094 & 3.957783 & 1.143778 \\
\hline
\end{tabular}
\label{tab:err1}
\end{table}

\begin{table}[H]
\centering
\caption{ Convergence-rates using approach \#1 and second order polynomials. }
\vspace{3mm}
\begin{tabular}{|l|l|l|l|l|l|l|l|l|l|l|}
\hline
$\lambda \backslash$N & $r_1$ & $r_2$ & $r_3$ \\
\hline
1 & 1.921346 & 1.979257 & 1.994746 \\
\hline
10 & 0.070003 & 4.701297 & 2.062245 \\
\hline
100 & 2.629290 & -0.102715 & -1.892538 \\
\hline
1000 & 1.343039 & -0.501525 & 1.180714 \\
\hline
10000 & -0.236797 & -1.235292 & 1.790886 \\
\hline
\end{tabular}
\label{tab:con1}
\end{table}


So what is happening? 

We can definitely see that when $\lambda$ gets bigger the numerical solution is bad. Only for $\lambda=1$ are the results reasonable. The convergence rates are in this case stable, but lower than expected. If we plot and compare the solutions we see that the deformations/displacements computed numerically are less then they should be. One might even say that the deformation seems to be \emph{"locked"}, see figure \ref{fig:locking}.

\begin{figure}[h!] 
\begin{center}
  \includegraphics[scale=0.4]{locking.png}
  \end{center}
  \caption{Comparing analytical and numerical solution for approach \#1.}
  \label{fig:locking}
\end{figure}

This is because we use elements that approximate the divergence poorly and when that term of the equation starts to dominate things fall apart. This phenomenon is called \emph{locking.} 

\textbf{Approach \#2}

Now, let's integrate the bad term by parts and see what happens. The variational formulation now reads:

\begin{framed} 
\texttt{a = inner(grad(u),grad(v))*dx+lmbda*inner(div(u),div(v))*dx \\
L = inner(f,v)*dx}
\end{framed}

\begin{table}[H]
\centering
\caption{ L2-errors from using approach \#2 and second order polynomials. }
\vspace{3mm}
\begin{tabular}{|l|l|l|l|l|l|l|l|l|l|l|}
\hline
$\lambda \backslash$N & 8 & 16 & 32 & 64 \\
\hline
1 & 2.080473e-03 & 2.52145e-04 & 3.12446e-05 & 3.89691e-06 \\
\hline
10 & 3.510135e-03 & 3.24355e-04 & 3.40039e-05 & 3.98917e-06 \\
\hline
100 & 1.439727e-02 & 1.49814e-03 & 1.19452e-04 & 8.74370e-06 \\
\hline
1000 & 2.699475e-02 & 5.14354e-03 & 6.89956e-04 & 6.37015e-05 \\
\hline
10000 & 2.989324e-02 & 7.17495e-03 & 1.57723e-03 & 2.72196e-04 \\
\hline
\end{tabular}
\end{table}

\begin{figure}[h!] 
\begin{center}
  \includegraphics[scale=0.25]{LE1.png}
  \end{center}
  \caption{Log-log plot of error for first and second order polynomials for approach \#2.}
\end{figure}


\begin{table}[H]
\centering
\caption{Convergence-rates using approach \#2 and second order polynomials. }
\vspace{3mm}
\begin{tabular}{|l|l|l|l|l|l|l|l|l|l|l|}
\hline
$\lambda \backslash$N & $r_1$ & $r_2$ & $r_3$ \\
\hline
1 & 3.044586 & 3.01258 & 3.00320 \\
\hline
10 & 3.435881 & 3.25380 & 3.09154 \\
\hline
100 & 3.264555 & 3.64867 & 3.77204 \\
\hline
1000 & 2.391844 & 2.89819 & 3.43711 \\
\hline
10000 & 2.058779 & 2.18558 & 2.53468 \\
\hline
\end{tabular}
\end{table}

The solution has now improved, but that are still some problems. For $\lambda=1$ everything works fine, the error decreases, the convergence rates are stable and of the expected order. For $\lambda \geq 10$ we can see fluctuations in the convergence rate, even though the $L_2-$errors are fairly low. 

Next; let's try to be really clever!

\textbf{Approach \#3: Stabilization using a mixed finite element formulation}

We define $p=\lambda(\nabla \cdot \mathbf{u})$ such that $\lambda \nabla \nabla \cdot \mathbf{u} = \nabla p$. Problem (2.1)-(2.2) can then be formulated
\begin{align*}
-\mu \Delta \mathbf{u} - \nabla p = \mathbf{f}, \\ 
(\nabla \cdot \mathbf{u}) - \frac{1}{\lambda} p, = 0
\end{align*}
with the weak formulation 
\begin{align*}
\int_\Omega \nabla \mathbf{u} : \nabla \mathbf{v} dx + \int_\Omega p(\nabla \cdot \mathbf{v}) dx = \int_\Omega \mathbf{f} \cdot \mathbf{v} dx \quad \forall \mathbf{v}\in V,  \\
\int_\Omega (\nabla \cdot \mathbf{u})q dx - \frac{1}{\lambda} \int_{\Omega} pq dx = 0 \quad \forall q \in Q. 
\end{align*}

We implement this in FEniCS and look at the results with a happy face. 

\begin{framed}
\texttt{	a = inner(grad(u),grad(v))*dx + div(v)*p*dx \\
	a2 = q*div(u)*dx - (1.0/lmbda)*p*q*dx \\
	 L = inner(f,v)*dx  }
\end{framed}

\begin{table}[H]
\centering
\caption{ L2-errors using a mixed FEM formulation $P_2 -P_1$. }
\vspace{3mm}
\begin{tabular}{|l|l|l|l|l|l|l|l|l|l|l|}
\hline
$\lambda \backslash$N & 8 & 16 & 32 & 64 \\
\hline
1 & 1.987451e-03 & 2.48875e-04 & 3.11388e-05 & 3.89358e-06 \\
\hline
10 & 1.968511e-03 & 2.48207e-04 & 3.11171e-05 & 3.89289e-06 \\
\hline
100 & 1.971773e-03 & 2.48317e-04 & 3.11211e-05 & 3.89304e-06 \\
\hline
1000 & 1.973119e-03 & 2.48367e-04 & 3.11230e-05 & 3.89311e-06 \\
\hline
10000 & 1.973277e-03 & 2.48373e-04 & 3.11233e-05 & 3.89312e-06 \\
\hline
\end{tabular}
\end{table}

\begin{table}[H]
\centering
\caption{ Convergence-rates using a mixed FEM formulation $P_2 -P_1$.}
\vspace{3mm}
\begin{tabular}{|l|l|l|l|l|l|l|l|l|l|l|}
\hline
$\lambda \backslash$N & $r_1$ & $r_2$ & $r_3$ \\
\hline
1 & 2.997423 & 2.99864 & 2.99955 \\
\hline
10 & 2.987488 & 2.99577 & 2.99879 \\
\hline
100 & 2.989240 & 2.99621 & 2.99892 \\
\hline
1000 & 2.989934 & 2.99642 & 2.99899 \\
\hline
10000 & 2.990015 & 2.99644 & 2.99899 \\
\hline
\end{tabular}
\end{table}

Our results are now very satisfactory. The numerical solution is now unaffected by the size of $\lambda$. We have successfully avoided \emph{locking}.  


\end{document}
